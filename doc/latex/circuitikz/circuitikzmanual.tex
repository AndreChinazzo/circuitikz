\documentclass[a4paper]{article}
\usepackage{ifxetex}
%for xelatex:


\ifxetex
	%compiler is xetex
	\usepackage{fontspec}
	\defaultfontfeatures{Ligatures=TeX, Numbers=OldStyle,Mapping=tex-text ,SmallCapsFeatures={LetterSpace=8, Numbers=OldStyle}}
	\setmainfont
	% {Linux Libertine}
	{Gentium Book Basic} 
\else
	%for pdflatex:
	\usepackage[T1]{fontenc}
	\usepackage[utf8]{inputenc}
	\usepackage[ngerman]{babel}
\fi






\usepackage{microtype}
	
%\usepackage[siunitx]{./../../../tex/latex/circuitikz/circuitikz}
\usepackage[siunitx]{../../../circuitikz_merged}

\sisetup{load=derived} % loading \siemens
\usepackage{showexpl}
\lstset{pos=l,width=-99pt, overhang=0pt,hsep=\columnsep,vsep=\bigskipamount,rframe=single,numbers=left,numberstyle=\tiny,numbersep=.3em, xleftmargin=1em, columns=flexible, language=[LaTeX]TEX}

\usepackage{framed}
\usepackage{hyperref}
\hypersetup{
    bookmarks=false,         % show bookmarks bar?
    pdftitle={CircuitTikZ v. \pgfcircversion\ - manual},    % title
    pdfauthor={Massimo Redaelli},     % author
    pdfsubject={CircuitTikZ manual},   % subject of the document
    pdfkeywords={}, % list of keywords
    colorlinks=true,       % false: boxed links; true: colored links
    linkcolor=black,          % color of internal links
    citecolor=black,        % color of links to bibliography
    filecolor=black,      % color of file links
    urlcolor=black           % color of external links
}
\usepackage{imakeidx}
\usepackage{textcomp}
\makeindex[title=Index of the components, intoc=true]

\def\circuititem#1#2#3{\item {#2} (\texttt{#1}) \index{#1} \par \begin{center}\begin{circuitikz} \draw (0,0) node[#1] {}; \end{circuitikz} \end{center}
\par}

\newcommand{\circuititembip}[3]{\item {#2} \index{#1}
\tikz\foreach \i in {#3} {\index{\i|see{#1}} }; (\texttt{#1}%
\ifthenelse{\equal{#3}{}}{%
}{%
, or \texttt{#3}%
}%
)\par \begin{center}\begin{circuitikz} \draw (0,0) to[#1] (2,0); \end{circuitikz} \end{center}\par}

\usepackage{marvosym} 
\newcommand{\email}[2][]{\def\temp{#1}\ifx\temp\empty\Email~\fi\href{mailto:#2}{#2}}

\long\def\comment#1{}

\begin{document}
\setcounter{secnumdepth}{3}
\setcounter{tocdepth}{3} 

\def\TikZ{Ti\emph{k}Z}
\def\Circuitikz{Circui\TikZ}
\def\ConTeXt{Con\TeX t}
\lstset{frameround=fttt}
\lstloadlanguages{TeX}

\title{\Circuitikz \\{\large version \pgfcircversion}}
\author{Massimo A. Redaelli}
\date{\today}

\maketitle

\tableofcontents

\section{Introduction}
After two years of little exposure only on my personal website\footnote{Now the package is moved to its own git repository: \url{https://github.com/mredaelli/circuitikz}. Contributions are welcome.}, I did a major rehauling of the code of Circui\TikZ, fixing several problems and converting everything to \TikZ\ version $2.0$.

I'm not too sure about the result, because my (La)\TeX\ skills are much to be improved, but it seems it's time for more user feedback. So, here it is\ldots

\medskip

I know the documentation is somewhat scant. Hope to have time to improve it a bit.

\subsection{About}
This package provides a set of macros for naturally typesetting electrical and (somewhat less naturally, perhaps) electronical networks.

It was born mainly for writing my own exercise book and exams sheets for the Elettrotecnica courses at Politecnico di Milano, Italy. I wanted a tool that was easy to use, with a lean syntax, native to \LaTeX, and supporting directly PDF output format.

So I based everything with the very impressive (if somewhat verbose at times) \TikZ\ package.

\subsection{Loading the package}
\verb!\usepackage{circuitikz}!

\TikZ\ will be automatically loaded.

\subsection{License}
Copyright \copyright\ 2007--2011 Massimo Redaelli. This package is author-maintained.
Permission is granted to copy, distribute and/or modify this software under the
terms of the \LaTeX Project Public License, version 1.3.1, or the GNU Public License. This software is provided ‘as is’, without warranty of any kind, either expressed or implied, including, but not limited to, the implied warranties of merchantability and fitness for a
particular purpose.

\subsection{Feedback}
Much appreciated: \email{mailto:m.redaelli@gmail.com}. Although I don't guarantee quick answers.

\subsection{Requirements}
\begin{itemize}
 \item \texttt{tikz}, version $\ge 2$;
 \item \texttt{xstring}, not older than 2009/03/13;
 \item \texttt{siunitx}, if using \texttt{siunitx} option.
\end{itemize}

\subsection{Incompatible packages}

\TikZ's own \texttt{circuit} library, which is based on \Circuitikz, (re?)defines several styles used by this library. In order to have them work together I added the \texttt{compatibility} package option, which basically prefixes the names of all \Circuitikz\ \texttt{to[]} styles with an asterisk.

So, if loaded with said option, one must write \verb!(0,0) to[*R] (2,0)! and, for transistors on a path, \verb!(0,0) to[*Tnmos] (2,0)!, and so on (but \verb!(0,0) node [nmos] {}!). See example at page~\pageref{ex:compatibility}.

\subsection{Introduction to version 0.3.0}

Probably nobody is hoping or caring for a new version of the package at this point, seeing how long it took me for this next release. But here it is, fixing a big problem (voltage labels in the wrong place, in some cases) and adding several components.

Thanks for bug reporting and suggesting improvements.

\subsection{Introduction to version 0.2.3}

Having waited a long time before updating the package, many feature requests piled on my desk. They should all be implemented now.

There are a number of backward incompatibilities--I'm sorry, but I had to make a choice in order not to have a schizophrenic interface. They are mostly, in my opinion, minor problems that can be dealt with with appropriate package options:
\begin{itemize}
	\item \texttt{potentiometer} is now the standard resistor-with-arrow-in-the-middle; the old potentiometer is now known as \texttt{variable resistor} (or \texttt{vR}), similarly to \texttt{variable inductor} and \texttt{variable capacitor};
	\item \texttt{american inductor} was not really the standard american inductor. The old american inductor has been renamed \texttt{cute inductor};
	\item \texttt{transformer}, \texttt{transformer core} and \texttt{variable inductor} are now linked with the chosen type of \texttt{inductor};
	\item styles for selecting shape variants (like \texttt{[american resistors]}) are now in the plural to avoid conflict with paths (like \texttt{to[american resistor]}).
\end{itemize}

\subsection{\ConTeXt\ compatibility}

As requested by some users, I fixed the package for it to be compatible with \ConTeXt. Just use \verb!\usemodule[circuitikz]! in your preamble and include the code between \verb!\startcircuitikz! and \verb!\endcircuitikz!. Please notice that the package \texttt{siunitx} in \emph{not} available for \ConTeXt: the option \texttt{siunitx} simply defines a few measurement units typical in electric sciences.

\medskip

In actually using \Circuitikz\ with \TikZ\ version 2 in \ConTeXt\ an error comes up, saying something like
\begin{verbatim}
! Undefined control sequence.
\tikz@cc@mid@checks -> \pgfutil@ifnextchar!
\end{verbatim}


The solution has been suggested to me by Aditya Mahajan, and involves modifying a file in \TikZ:
\begin{verbatim}
Here is the fix. In tikzlibrarycalc.code.tex change

\def\tikz@cc@mid@checks{
   \pgfutil@ifnextchar !{%AM: Added space
     \tikz@cc@mid%
   }{%
     \advance\pgf@xa by\tikz@cc@factor\pgf@xb%
     \advance\pgf@ya by\tikz@cc@factor\pgf@yb%
     \tikz@cc@parse% continue
   }%
}

\def\tikz@cc@mid !{%AM Added space
   \pgfutil@ifnextchar({%
     \tikz@scan@one@point\tikz@cc@project%
   }{%
     \tikz@cc@mid@num%
   }%
}
\end{verbatim}

As far as I know, this is a bug in \TikZ, and I notified the author, but until he fixes it, you know the workaround.


\section{Options}

\begin{itemize}
	\item \texttt{europeanvoltages}: uses arrows to define voltages, and uses european-style voltage sources;
	\item \texttt{americanvoltages}: uses $-$ and $+$ to define voltages, and uses american-style voltage sources;
	\item \texttt{europeancurrents}: uses european-style current sources;
	\item \texttt{americancurrents}: uses american-style current sources;
	\item \texttt{europeanresistors}: uses rectangular empty shape for resistors, as per european standards;
	\item \texttt{americanresistors}: uses zig-zag shape for resistors, as per american standards;
	\item \texttt{europeaninductors}: uses rectangular filled shape for inductors, as per european standards;
	\item \texttt{americaninductors}: uses "4-bumps" shape for inductors, as per american standards;
	\item \texttt{cuteinductors}: uses my personal favorite, "pig-tailed" shape for inductors;
	\item \texttt{americanports}: uses triangular logic ports, as per american standards;
	\item \texttt{europeanports}: uses rectangular logic ports, as per european standards;
	\item \texttt{americangfsurgearrester}: uses round gas filled surge arresters, as per american standards;
	\item \texttt{europeangfsurgearrester}: uses rectangular gas filled surge arresters, as per european standards;
	\item \texttt{european}: equivalent to \texttt{europeancurrents}, \texttt{europeanvoltages}, \texttt{europeanresistors}, \texttt{europeaninductors}, \texttt{europeanports}, \texttt{europeangfsurgearrester};
	\item \texttt{american}: equivalent to \texttt{americancurrents}, \texttt{americanvoltages}, \texttt{americanresistors}, \texttt{americaninductors}, \texttt{americanports}, \texttt{americangfsurgearrester};
	\item \texttt{siunitx}: integrates with \texttt{SIunitx} package. If labels, currents or voltages are of the form \verb!#1<#2>! then what is shown is actually \verb!\SI{#1}{#2}!; 
	\item \texttt{nosiunitx}: labels are not interpreted as above;
	\item \texttt{fulldiodes}: the various diodes are drawn \emph{and} filled by default, i.e. when using styles such as \texttt{diode}, \texttt{D}, \texttt{sD}, \ldots Un-filled diode can always be forced with \texttt{Do}, \texttt{sDo}, \ldots
	\item \texttt{emptydiodes}: the various diodes are drawn \emph{but not} filled by default, i.e. when using styles such as \texttt{diode}, \texttt{D}, \texttt{sD}, \ldots Filled diode can always be forced with \texttt{D*}, \texttt{sD*}, \ldots
	\item \texttt{arrowmos}: pmos and nmos have arrows analogous to those of pnp and npn transistors;
	\item \texttt{noarrowmos}: pmos and nmos do not have arrows analogous to those of pnp and npn transistors;
	\item \texttt{straightlabels}: labels on bipoles are always printed straight up, i.e.~with horizontal baseline;
	\item \texttt{rotatelabels}: labels on bipoles are always printed aligned along the bipole;
	\item \texttt{smartlabels}: labels on bipoles are rotated along the bipoles, unless the rotation is very close to multiples of 90°;
	\item \texttt{compatibility}: makes it possibile to load \Circuitikz\ and \TikZ\ circuit library together.
\end{itemize}	

The old options in the singular (like \texttt{american voltage}) are still available for compatibility, but are discouraged.

\medskip

Loading the package with no options is equivalent to my own personal liking, that is to the following options:\\
 \texttt{[europeancurrents, europeanvoltages, americanresistors, cuteinductors, americanports, nosiunitx, noarrowmos, smartlabels]}.
 
\medskip

In \ConTeXt\ the options are similarly specified: \texttt{current=european|american}, \texttt{voltage=european|american},  \texttt{resistor=american|european},  \texttt{inductor=cute|american|european}, \texttt{logic=american|european}, \texttt{siunitx=true|false}, \texttt{arrowmos=false|true}.
 
\section{The components}

Here follows the list of all the shapes defined by Circui\TikZ. These are all \texttt{pgf} nodes, so they are usable in both \texttt{pgf} and \TikZ.

\medskip

Each bipole (plus triac and thyristors) are shown using the following command, where \verb!#1! is the name of the component\footnote{If \texttt{\#1} is the name of the bipole/the style, then the actual name of the shape is \texttt{\#1shape}.}:

\begin{verbatim}
\begin{center}\begin{circuitikz} \draw 
   (0,0) to[ #1 ] (2,0)
; \end{circuitikz} \end{center}
\end{verbatim}

The other shapes are shown with:
\begin{verbatim}
\begin{center}\begin{circuitikz} \draw 
   (0,0) node[ #1 ] {}
; \end{circuitikz} \end{center}
\end{verbatim}

Please notice that for user convenience transistors can also be inputted using the syntax for bipoles. See section~\ref{sec:transasbip}.

\begin{framed}
	If using the \verb!\tikzexternalize! feature, as of Ti\emph{k}z 2.1 all pictures must end with \verb!\end{tikzpicture}!. Thus you \emph{cannot} use the \verb!circuitikz! environment.
	
	Which is ok: just use \verb!tikzpicture!: everything will work there just fine.
\end{framed}

\subsection{Monopoles}
\begin{itemize}
	\circuititem{ground}{Ground}{}
	\circuititem{rground}{Reference ground}{}
	\circuititem{sground}{Signal ground}{}
	\circuititem{nground}{Noiseless ground}{}
	\circuititem{pground}{Protective ground}{}
	\circuititem{cground}{Chassis ground\footnote{These last three were contributed by Luigi «Liverpool»)}}{}
	\circuititem{antenna}{Antenna}{}
	\circuititem{rxantenna}{Receiving antenna}{}
	\circuititem{txantenna}{Transmitting antenna}{}
	\circuititem{tlinestub}{Transmission line stub}{}
\end{itemize}


\subsection{Bipoles}

\subsubsection{Instruments}
\begin{itemize}
	\circuititembip{ammeter}{Ammeter}{}
	\circuititembip{voltmeter}{Voltmeter}{}
	\circuititembip{ohmmeter}{Ohmmeter}{}
\end{itemize}	

\subsubsection{Basic resistive bipoles}
\begin{itemize}
	\circuititembip{short}{Short circuit}{}
	\circuititembip{open}{Open circuit}{}
	
	\circuititembip{lamp}{Lamp}{}
	\circuititembip{generic}{Generic (symmetric) bipole}{}
	\circuititembip{tgeneric}{Tunable generic bipole}{}
	\circuititembip{ageneric}{Generic asymmetric bipole}{}
	\circuititembip{fullgeneric}{Generic asymmetric bipole (full)}{}
	\circuititembip{tfullgeneric}{Tunable generic  bipole (full)}{}
	\circuititembip{memristor}{Memristor}{Mr}
\end{itemize}	

\subsubsection{Resistors and the like}

If (default behaviour) \texttt{americanresistors} option is active (or the style \texttt{[american resistors]} is used), the resistor is displayed as follows:
\begin{itemize}
  	\ctikzset{resistor=american}
  	\circuititembip{R}{Resistor}{american resistor}
	\circuititembip{vR}{Variable resistor}{variable american resistor}
	\circuititembip{pR}{Potentiometer}{american potentiometer}
\end{itemize}

If  instead \texttt{europeanresistors} option is active (or the style \texttt{[european resistors]} is used), the resistors, variable resistors and potentiometers are displayed as follows:
\begin{itemize}
  	\ctikzset{resistor=european}
  	\circuititembip{R}{Resistor}{european resistor}
	\circuititembip{vR}{Variable resistor}{european variable resistor}
	\circuititembip{pR}{Potentiometer}{european potentiometer}
	\ctikzset{resistor=american} % reset default
\end{itemize}

Other miscellaneous resistor-like devices:
\begin{itemize}
  	\circuititembip{varistor}{Varistor}{}
	\circuititembip{phR}{Photoresistor}{photoresistor}
	\circuititembip{thermocouple}{Thermocouple}{}
	\circuititembip{thR}{Thermistor}{thermistor}
	\circuititembip{thRp}{PTC thermistor}{thermistor ptc}
	\circuititembip{thRn}{NTC thermistor}{thermistor ntc}
	\circuititembip{fuse}{Fuse}{}
	\circuititembip{afuse}{Asymmetric fuse}{asymmetric fuse}
\end{itemize}

\subsubsection{Stationary sources}
\begin{itemize}
	\circuititembip{battery}{Battery}{}
	\circuititembip{battery1}{Single battery cell}{}
	\circuititembip{european voltage source}{Voltage source (european style)}{}
	\circuititembip{american voltage source}{Voltage source (american style)}{}
	\circuititembip{european current source}{Current source (european style)}{}
	\circuititembip{american current source}{Current source (american style)}{}
\end{itemize}

\begin{framed}
If (default behaviour) \texttt{europeancurrents} option is active (or the style \texttt{[european currents]} is used), the shorthands \texttt{current source}, \texttt{isource}, and \texttt{I} are equivalent to \texttt{european current source}. Otherwise, if \texttt{americancurrents} option is active (or the style \texttt{[american currents]} is used) they are equivalent to \texttt{american current source}.

Similarly, if (default behaviour) \texttt{europeanvoltages} option is active (or the style \texttt{[european voltages]} is used), the shorthands \texttt{voltage source}, \texttt{vsource}, and \texttt{V} are equivalent to \texttt{european voltage source}. Otherwise, if \texttt{americanvoltages} option is active (or the style \texttt{[american voltages]} is used) they are equivalent to \texttt{american voltage source}.
\end{framed}



\subsubsection{Diodes and such}
\begin{itemize}
	\circuititembip{empty diode}{Empty diode}{Do}
	\circuititembip{empty Schottky diode}{Empty Schottky diode}{sDo}
	\circuititembip{empty Zener diode}{Empty Zener diode}{zDo}
	\circuititembip{empty tunnel diode}{Empty tunnel diode}{tDo}
	\circuititembip{empty photodiode}{Empty photodiode}{pDo}
	\circuititembip{empty led}{Empty led}{leDo}
	\circuititembip{empty varcap}{Empty varcap}{VCo}
	\circuititembip{full diode}{Full diode}{D*}
	\circuititembip{full Schottky diode}{Full Schottky diode}{sD*}
	\circuititembip{full Zener diode}{Full Zener diode}{zD*}
	\circuititembip{full tunnel diode}{Full tunnel diode}{tD*}
	\circuititembip{full photodiode}{Full photodiode}{pD*}
	\circuititembip{full led}{Full led}{leD*}
	\circuititembip{full varcap}{Full varcap}{VC*}
	\end{itemize}

\begin{framed}
The options \texttt{fulldiodes} and \texttt{emptydiodes} (and the styles \texttt{[full diodes]} and \texttt{[empty diodes]}) define which shape will be used by abbreviated commands such that \texttt{D}, \texttt{sD}, \texttt{zD}, \texttt{tD}, \texttt{pD}, \texttt{leD}, and \texttt{VC}.
\end{framed}

\begin{itemize}
	\circuititembip{squid}{Squid}{}
	\circuititembip{barrier}{Barrier}{}
\end{itemize}

\begin{itemize}
	\circuititembip{european gas filled surge arrester}{European gas filled surge arrester}{}
	\circuititembip{american gas filled surge arrester}{American gas filled surge arrester}{}
\end{itemize}

\begin{framed}
If (default behaviour) \texttt{europeangfsurgearrester} option is active (or the style \texttt{[european gas filled surge arrester]} is used), the shorthands \texttt{gas filled surge arrester} and \texttt{gf surge arrester} are equivalent to the european version of the component.

If otherwise \texttt{americangfsurgearrester} option is active (or the style \texttt{[american gas filled surge arrester]} is used), the shorthands the shorthands \texttt{gas filled surge arrester} and \texttt{gf surge arrester} are equivalent to the american version of the component.
\end{framed}

\subsubsection{Basic dynamical bipoles}
\begin{itemize}
	\circuititembip{capacitor}{Capacitor}{C}
	\circuititembip{polar capacitor}{Polar capacitor}{pC}
	\circuititembip{variable capacitor}{Variable capacitor}{vC}
\end{itemize}	

If (default behaviour) \texttt{cuteinductors} option is active (or the style \texttt{[cute inductors]} is used), the inductors are displayed as follows:
\begin{itemize}
  	\ctikzset{inductor=cute}
  	\circuititembip{L}{Inductor}{cute inductor}
	\circuititembip{vL}{Variable inductor}{variable cute inductor}
\end{itemize}

If \texttt{americaninductors} option is active (or the style \texttt{[american inductors]} is used), the inductors are displayed as follows:
\begin{itemize}
  	\ctikzset{inductor=american}
  	\circuititembip{L}{Inductor}{american inductor}
	\circuititembip{vL}{Variable inductor}{variable american inductor}
\end{itemize}

Finally, if \texttt{europeaninductors} option is active (or the style \texttt{[european inductors]} is used), the inductors are displayed as follows:
\begin{itemize}
  	\ctikzset{inductor=european}
  	\circuititembip{L}{Inductor}{european inductor}
	\circuititembip{vL}{Variable inductor}{variable european inductor}
\end{itemize}

There is also a transmission line: 
\begin{itemize}
\circuititembip{TL}{Transmission line}{transmission line, tline}
\end{itemize}

\subsubsection{Sinusoidal sources} Here because I was asked for them. But how do you distinguish one from the other?!
\begin{itemize}
	\circuititembip{sinusoidal voltage source}{Sinusoidal voltage source}{vsourcesin, sV}
	\circuititembip{sinusoidal current source}{Sinusoidal current source}{isourcesin, sI}
\end{itemize}

\subsubsection{Special sources}
\begin{itemize}
	\circuititembip{square voltage source}{Square voltage source}{vsourcesquare, sqV}
	\circuititembip{vsourcetri}{Square voltage source}{}
	\circuititembip{esource}{Empty voltage source}{}
\end{itemize}

\subsubsection{DC sources}
\begin{itemize}
	\circuititembip{dcvsource}{DC voltage source}{}
	\circuititembip{dcisource}{DC current source}{}
\end{itemize}

%\begin{framed}
%The options \texttt{europeancurrent} [resp. \texttt{europeanvoltage}] (the default) and \texttt{americancurrent} [resp. \texttt{americanvoltage}] define which sinusoidal current [resp. voltage] source is selected by default when the abbreviated styles \texttt{sinusoidal current source}, \texttt{csourcesin}, \texttt{cI} [resp. \texttt{sinusoidal voltage source}, \texttt{vsourcesin}, \texttt{cV}] are used.

%One can also use the related styles \texttt{[european currents]} [resp. \texttt{[european voltages]}] and \texttt{[american currents]} [resp. \texttt{[american voltages]}].
%\end{framed}

\subsubsection{Switch}
\begin{itemize}
	\circuititembip{closing switch}{Closing switch}{cspst}
	\circuititembip{opening switch}{Opening switch}{ospst}
	\circuititembip{push button}{Push button}{}
\end{itemize}	



\subsection{Tripoles}
\subsubsection{Controlled sources} Admittedly, graphically they are bipoles. But I couldn't\ldots
\begin{itemize}
	\circuititembip{european controlled voltage source}{Controlled voltage source (european style)}{}
	\circuititembip{american controlled voltage source}{Controlled voltage source (american style)}{}
	\circuititembip{european controlled current source}{Controlled current source (european style)}{}
	\circuititembip{american controlled current source}{Controlled current source (american style)}{}
\end{itemize}

\begin{framed}
If (default behaviour) \texttt{europeancurrents} option is active (or the style \texttt{[european currents]} is used), the shorthands \texttt{controlled current source}, \texttt{cisource}, and \texttt{cI} are equivalent to \texttt{european controlled current source}. Otherwise, if \texttt{americancurrents} option is active (or the style \texttt{[american currents]} is used) they are equivalent to \texttt{american controlled current source}.

Similarly, if (default behaviour) \texttt{europeanvoltages} option is active (or the style \texttt{[european voltages]} is used), the shorthands \texttt{controlled voltage source}, \texttt{cvsource}, and \texttt{cV} are equivalent to \texttt{european controlled voltage source}. Otherwise, if \texttt{americanvoltages} option is active (or the style \texttt{[american voltages]} is used) they are equivalent to \texttt{american controlled voltage source}.
\end{framed}

\begin{itemize}
	\circuititembip{controlled sinusoidal voltage source}{Controlled sinusoidal voltage source}{controlled vsourcesin, cvsourcesin, csV}
	\circuititembip{controlled sinusoidal current source}{Controlled sinusoidal current source}{controlled isourcesin, cisourcesin, csI}
	\end{itemize}


\subsubsection{Transistors} 
\begin{itemize}
	\circuititem{nmos}{\scshape nmos}{}
	\circuititem{pmos}{\scshape pmos}{}
	\circuititem{npn}{\scshape npn}{}
	\circuititem{pnp}{\scshape pnp}{}
	\circuititem{nigbt}{\scshape npigbt}{}
	\circuititem{pigbt}{\scshape pigbt}{}	
\end{itemize}

If the option \texttt{arrowmos} is used (or after the commant \verb!\ctikzset{tripoles/mos style/arrows}! is given), this is the output:
\ctikzset{tripoles/mos style/arrows}
\begin{itemize}
	\circuititem{nmos}{\scshape nmos}{}
	\circuititem{pmos}{\scshape pmos}{}
\end{itemize}
\ctikzset{tripoles/mos style/no arrows}

\textsc{nfet}s and \textsc{pfet}s have been incorporated based on code provided by Clemens Helfmeier and Theodor 
Borsche:
\begin{itemize}
	\circuititem{nfet}{\scshape nfet}{}
	\circuititem{nigfete}{\scshape nigfete}{}
	\circuititem{nigfetebulk}{\scshape nigfetebulk}{}
	\circuititem{nigfetd}{\scshape nigfetd}{}
	\circuititem{pfet}{\scshape pfet}{}
	\circuititem{pigfete}{\scshape pigfete}{}
	\circuititem{pigfetebulk}{\scshape pigfetebulk}{}
	\circuititem{pigfetd}{\scshape pigfetd}{}
\end{itemize}

\textsc{njfet} and \textsc{pjfet} have been incorporated based on code provided by Danilo Piazzalunga: 
\begin{itemize}
	\circuititem{njfet}{\scshape njfet}{}
	\circuititem{pjfet}{\scshape pjfet}{}
\end{itemize}

\textsc{isfet}
\begin{itemize}
	\circuititem{isfet}{\scshape isfet}{}
\end{itemize}

\subsubsection{Switch}
\begin{itemize}
	\circuititem{spdt}{\scshape spdt}{}
	\circuititembip{toggle switch}{Toggle switch}{}
\end{itemize}

\subsubsection{Other bipole-like tripoles}\label{sec:othertrip} The following tripoles are entered with the usual command of the form 
\begin{itemize}
	\circuititembip{triac}{triac}{Tr}
	\circuititembip{thyristor}{thyristor}{Ty}
	%\circuititembip{IGBT}{IGBT}{}
\end{itemize}

\subsubsection{Misc}
\begin{itemize}
	\circuititem{mixer}{Mixer}{}
\end{itemize}

\subsection{Double bipoles}

Transformers automatically use the inductor shape currently selected. These are the three possibilities:
\begin{itemize}
	\ctikzset{inductor=cute}
	\circuititem{transformer}{Transformer (cute inductor)}{}
	\ctikzset{inductor=american}
	\circuititem{transformer}{Transformer (american inductor)}{}
	\ctikzset{inductor=european}
	\circuititem{transformer}{Transformer (european inductor)}{}
\end{itemize}


Transformers with core are also available:
\begin{itemize}
	\ctikzset{inductor=cute}
	\circuititem{transformer core}{Transformer  core (cute inductor)}{}
	\ctikzset{inductor=american}
	\circuititem{transformer core}{Transformer core (american inductor)}{}
	\ctikzset{inductor=european}
	\circuititem{transformer core}{Transformer core (european inductor)}{}
	\ctikzset{inductor=cute} % reset default
\end{itemize}

\begin{itemize}
	\circuititem{gyrator}{Gyrator}{}
\end{itemize}


\subsection{Logic gates}

\begin{itemize}
	\circuititem{american and port}{American \textsc{and} port}{}
	\circuititem{american or port}{American \textsc{or} port}{}
	\circuititem{american not port}{American \textsc{not} port}{}
	\circuititem{american nand port}{American \textsc{nand} port}{}
	\circuititem{american nor port}{American \textsc{nor} port}{}
	\circuititem{american xor port}{American \textsc{xor} port}{}
	\circuititem{american xnor port}{American \textsc{xnor} port}{}
\end{itemize}

\begin{itemize}
	\circuititem{european and port}{European \textsc{and} port}{}
	\circuititem{european or port}{European \textsc{or} port}{}
	\circuititem{european not port}{European \textsc{not} port}{}
	\circuititem{european nand port}{European \textsc{nand} port}{}
	\circuititem{european nor port}{European \textsc{nor} port}{}
	\circuititem{european xor port}{European \textsc{xor} port}{}
	\circuititem{european xnor port}{European \textsc{xnor} port}{}
\end{itemize}

\begin{framed}
If (default behaviour) \texttt{americanports} option is active (or the style \texttt{[american ports]} is used), the shorthands \texttt{and port}, \texttt{or port}, \texttt{not port}, \texttt{nand port}, \texttt{not port}, \texttt{xor port}, and \texttt{xnor port} are equivalent to the american version of the respective logic port.

If otherwise \texttt{europeanports} option is active (or the style \texttt{[european ports]} is used), the shorthands \texttt{and port}, \texttt{or port}, \texttt{not port}, \texttt{nand port}, \texttt{not port}, \texttt{xor port}, and \texttt{xnor port} are equivalent to the european version of the respective logic port.
\end{framed}

\subsection{Amplifiers}

\begin{itemize}
	\circuititem{op amp}{Operational amplifier}{}
	\circuititem{fd op amp}{Fully differential operational amplifier\footnote{Contributed by Kristofer M. Monisit.}}{}
	\circuititem{gm amp}{transconductance amplifier}{}
	\circuititem{plain amp}{Plain amplifier}{}
	\circuititem{buffer}{Buffer}{}
\end{itemize}

\subsection{Support shapes}

\begin{itemize}
	\circuititem{currarrow}{Arrows (current and voltage)}{}
	\circuititem{circ}{Connected terminal}{}
	\circuititem{ocirc}{Unconnected terminal}{}
\end{itemize}



\section{Usage}

\begin{LTXexample}[varwidth=true]
\begin{circuitikz}
   \draw (0,0) to[R, l=$R_1$] (2,0);
\end{circuitikz}
\end{LTXexample}	

\begin{LTXexample}[varwidth=true]
\begin{circuitikz}
   \draw (0,0) to[R=$R_1$] (2,0);
\end{circuitikz}
\end{LTXexample}	


\begin{LTXexample}[varwidth=true]
\begin{circuitikz}
   \draw (0,0) to[R, i=$i_1$] (2,0);
\end{circuitikz}
\end{LTXexample}	

\begin{LTXexample}[varwidth=true]
\begin{circuitikz}
   \draw (0,0) to[R, v=$v_1$] (2,0);
\end{circuitikz}
\end{LTXexample}	

\begin{LTXexample}[varwidth=true]
\begin{circuitikz}
   \draw (0,0) to[R=$R_1$, i=$i_1$, v=$v_1$] (2,0);
\end{circuitikz}
\end{LTXexample}	

\begin{LTXexample}[varwidth=true]
\begin{circuitikz}
   \draw (0,0) to[R=$R_1$, i=$i_1$, v=$v_1$] (2,0);
\end{circuitikz}
\end{LTXexample}


Long names/styles for the bipoles can be used:
\begin{LTXexample}[varwidth=true]
\begin{circuitikz}\draw
  (0,0) to[resistor=1<\kilo\ohm>] (2,0) 
;\end{circuitikz}
\end{LTXexample}

\subsection{Labels}

\begin{LTXexample}[varwidth=true]
\begin{circuitikz}
   \draw (0,0) to[R, l^=$R_1$] (2,0);
\end{circuitikz}
\end{LTXexample}	

\begin{LTXexample}[varwidth=true]
\begin{circuitikz}
   \draw (0,0) to[R, l_=$R_1$] (2,0);
\end{circuitikz}
\end{LTXexample}	

\noindent The default orientation of labels is controlled by the options \texttt{smartlabels}, \texttt{rotatelabels} and \texttt{straightlabels} (or the corresponding \texttt{label/align} keys). Here are examples to see the differences:
\begin{LTXexample}[varwidth=true]
\begin{circuitikz}
\ctikzset{label/align = straight}
\def\DIR{0,45,90,135,180,-90,-45,-135}
\foreach \i in \DIR {
  \draw (0,0) to[R=\i, *-o] (\i:2.5);
}
\end{circuitikz}
\end{LTXexample}	
\begin{LTXexample}[varwidth=true]
\begin{circuitikz}
\ctikzset{label/align = rotate}
\def\DIR{0,45,90,135,180,-90,-45,-135}
\foreach \i in \DIR {
  \draw (0,0) to[R=\i, *-o] (\i:2.5);
}
\end{circuitikz}
\end{LTXexample}	
\begin{LTXexample}[varwidth=true]
\begin{circuitikz}
\ctikzset{label/align = smart}
\def\DIR{0,45,90,135,180,-90,-45,-135}
\foreach \i in \DIR {
  \draw (0,0) to[R=\i, *-o] (\i:2.5);
}
\end{circuitikz}
\end{LTXexample}	

\subsection{Currents}

\begin{LTXexample}[varwidth=true]
\begin{circuitikz}
   \draw (0,0) to[R, i^>=$i_1$] (2,0);
\end{circuitikz}
\end{LTXexample}	

\begin{LTXexample}[varwidth=true]
\begin{circuitikz}
   \draw (0,0) to[R, i_>=$i_1$] (2,0);
\end{circuitikz}
\end{LTXexample}	

\begin{LTXexample}[varwidth=true]
\begin{circuitikz}
   \draw (0,0) to[R, i^<=$i_1$] (2,0);
\end{circuitikz}
\end{LTXexample}	

\begin{LTXexample}[varwidth=true]
\begin{circuitikz}
   \draw (0,0) to[R, i_<=$i_1$] (2,0);
\end{circuitikz}
\end{LTXexample}	

\begin{LTXexample}[varwidth=true]
\begin{circuitikz}
   \draw (0,0) to[R, i>^=$i_1$] (2,0);
\end{circuitikz}
\end{LTXexample}	

\begin{LTXexample}[varwidth=true]
\begin{circuitikz}
   \draw (0,0) to[R, i>_=$i_1$] (2,0);
\end{circuitikz}
\end{LTXexample}	

\begin{LTXexample}[varwidth=true]
\begin{circuitikz}
   \draw (0,0) to[R, i<^=$i_1$] (2,0);
\end{circuitikz}
\end{LTXexample}	

\begin{LTXexample}[varwidth=true]
\begin{circuitikz}
   \draw (0,0) to[R, i<_=$i_1$] (2,0);
\end{circuitikz}
\end{LTXexample}	

Also

\begin{LTXexample}[varwidth=true]
\begin{circuitikz}
   \draw (0,0) to[R, i<=$i_1$] (2,0);
\end{circuitikz}
\end{LTXexample}	

\begin{LTXexample}[varwidth=true]
\begin{circuitikz}
   \draw (0,0) to[R, i>=$i_1$] (2,0);
\end{circuitikz}
\end{LTXexample}	

\begin{LTXexample}[varwidth=true]
\begin{circuitikz}
   \draw (0,0) to[R, i^=$i_1$] (2,0);
\end{circuitikz}
\end{LTXexample}	

\begin{LTXexample}[varwidth=true]
\begin{circuitikz}
   \draw (0,0) to[R, i_=$i_1$] (2,0);
\end{circuitikz}
\end{LTXexample}	



\subsection{Voltages}

\subsubsection{European style} The default, with arrows. Use option \texttt{europeanvoltage} or style \verb![european voltages]!.

\begin{LTXexample}[varwidth=true]
\begin{circuitikz}[european voltages]
   \draw (0,0) to[R, v^>=$v_1$] (2,0);
\end{circuitikz}
\end{LTXexample}

\begin{LTXexample}[varwidth=true]
\begin{circuitikz}[european voltages]
   \draw (0,0) to[R, v^<=$v_1$] (2,0);
\end{circuitikz}
\end{LTXexample}

\begin{LTXexample}[varwidth=true]
\begin{circuitikz}[european voltages]
   \draw (0,0) to[R, v_>=$v_1$] (2,0);
\end{circuitikz}
\end{LTXexample}

\begin{LTXexample}[varwidth=true]
\begin{circuitikz}[european voltages]
   \draw (0,0) to[R, v_<=$v_1$] (2,0);
\end{circuitikz}
\end{LTXexample}

\subsubsection{American style} For those who like it (not me). Use option \texttt{americanvoltage} or set \verb![american voltages]!.

\begin{LTXexample}[varwidth=true]
\begin{circuitikz}[american voltages]
   \draw (0,0) to[R, v^>=$v_1$] (2,0);
\end{circuitikz}
\end{LTXexample}

\begin{LTXexample}[varwidth=true]
\begin{circuitikz}[american voltages]
   \draw (0,0) to[R, v^<=$v_1$] (2,0);
\end{circuitikz}
\end{LTXexample}

\begin{LTXexample}[varwidth=true]
\begin{circuitikz}[american voltages]
   \draw (0,0) to[R, v_>=$v_1$] (2,0);
\end{circuitikz}
\end{LTXexample}

\begin{LTXexample}[varwidth=true]
\begin{circuitikz}[american voltages]
   \draw (0,0) to[R, v_<=$v_1$] (2,0);
\end{circuitikz}
\end{LTXexample}



\subsection{Nodes}

\begin{LTXexample}[varwidth=true]
\begin{circuitikz}
   \draw (0,0) to[R, o-o] (2,0);
\end{circuitikz}
\end{LTXexample}

\begin{LTXexample}[varwidth=true]
\begin{circuitikz}
   \draw (0,0) to[R, -o] (2,0);
\end{circuitikz}
\end{LTXexample}

\begin{LTXexample}[varwidth=true]
\begin{circuitikz}
   \draw (0,0) to[R, o-] (2,0);
\end{circuitikz}
\end{LTXexample}

\begin{LTXexample}[varwidth=true]
\begin{circuitikz}
   \draw (0,0) to[R, *-*] (2,0);
\end{circuitikz}
\end{LTXexample}

\begin{LTXexample}[varwidth=true]
\begin{circuitikz}
   \draw (0,0) to[R, -*] (2,0);
\end{circuitikz}
\end{LTXexample}

\begin{LTXexample}[varwidth=true]
\begin{circuitikz}
   \draw (0,0) to[R, *-] (2,0);
\end{circuitikz}
\end{LTXexample}

\begin{LTXexample}[varwidth=true]
\begin{circuitikz}
   \draw (0,0) to[R, o-*] (2,0);
\end{circuitikz}
\end{LTXexample}

\begin{LTXexample}[varwidth=true]
\begin{circuitikz}
   \draw (0,0) to[R, *-o] (2,0);
\end{circuitikz}
\end{LTXexample}


\subsection{Special components}

For some components label, current and voltage behave as one would expect:

\begin{LTXexample}[varwidth=true]
\begin{circuitikz}
   \draw (0,0) to[I=$a_1$] (2,0);
\end{circuitikz}
\end{LTXexample}

\begin{LTXexample}[varwidth=true]
\begin{circuitikz}
   \draw (0,0) to[I, i=$a_1$] (2,0);
\end{circuitikz}
\end{LTXexample}


\begin{LTXexample}[varwidth=true]
\begin{circuitikz}
   \draw (0,0) to[cI=$k\cdot a_1$] (2,0);
\end{circuitikz}
\end{LTXexample}


\begin{LTXexample}[varwidth=true]
\begin{circuitikz}
   \draw (0,0) to[sI=$a_1$] (2,0);
\end{circuitikz}
\end{LTXexample}

\begin{LTXexample}[varwidth=true]
\begin{circuitikz}
   \draw (0,0) to[csI=$k\cdot a_1$] (2,0);
\end{circuitikz}
\end{LTXexample}

The following results from using the option \texttt{americancurrent} or using the style \verb![american currents]!.

\begin{LTXexample}[varwidth=true]
\begin{circuitikz}[american currents]
   \draw (0,0) to[I=$a_1$] (2,0);
\end{circuitikz}
\end{LTXexample}

\begin{LTXexample}[varwidth=true]
\begin{circuitikz}[american currents]
   \draw (0,0) to[I, i=$a_1$] (2,0);
\end{circuitikz}
\end{LTXexample}


\begin{LTXexample}[varwidth=true]
\begin{circuitikz}[american currents]
   \draw (0,0) to[cI=$k\cdot a_1$] (2,0);
\end{circuitikz}
\end{LTXexample}


\begin{LTXexample}[varwidth=true]
\begin{circuitikz}[american currents]
   \draw (0,0) to[sI=$a_1$] (2,0);
\end{circuitikz}
\end{LTXexample}

\begin{LTXexample}[varwidth=true]
\begin{circuitikz}[american currents]
   \draw (0,0) to[csI=$k\cdot a_1$] (2,0);
\end{circuitikz}
\end{LTXexample}

The same holds for voltage sources:

\begin{LTXexample}[varwidth=true]
\begin{circuitikz}
   \draw (0,0) to[V=$a_1$] (2,0);
\end{circuitikz}
\end{LTXexample}

\begin{LTXexample}[varwidth=true]
\begin{circuitikz}
   \draw (0,0) to[V, v=$a_1$] (2,0);
\end{circuitikz}
\end{LTXexample}


\begin{LTXexample}[varwidth=true]
\begin{circuitikz}
   \draw (0,0) to[cV=$k\cdot a_1$] (2,0);
\end{circuitikz}
\end{LTXexample}


\begin{LTXexample}[varwidth=true]
\begin{circuitikz}
   \draw (0,0) to[sV=$a_1$] (2,0);
\end{circuitikz}
\end{LTXexample}

\begin{LTXexample}[varwidth=true]
\begin{circuitikz}
   \draw (0,0) to[csV=$k\cdot a_1$] (2,0);
\end{circuitikz}
\end{LTXexample}

The following results from using the option \texttt{americanvoltage} or the style \verb![american voltages]!.

\begin{LTXexample}[varwidth=true]
\begin{circuitikz}[american voltages]
   \draw (0,0) to[V=$a_1$] (2,0);
\end{circuitikz}
\end{LTXexample}

\begin{LTXexample}[varwidth=true]
\begin{circuitikz}[american voltages]
   \draw (0,0) to[V, v=$a_1$] (2,0);
\end{circuitikz}
\end{LTXexample}


\begin{LTXexample}[varwidth=true]
\begin{circuitikz}[american voltages]
   \draw (0,0) to[cV=$k\cdot a_1$] (2,0);
\end{circuitikz}
\end{LTXexample}


\begin{LTXexample}[varwidth=true]
\begin{circuitikz}[american voltages]
   \draw (0,0) to[sV=$a_1$] (2,0);
\end{circuitikz}
\end{LTXexample}

\begin{LTXexample}[varwidth=true]
\begin{circuitikz}[american voltages]
   \draw (0,0) to[csV=$k\cdot a_1$] (2,0);
\end{circuitikz}
\end{LTXexample}

\subsection{Integration with {\ttfamily siunitx}}

If the option {\ttfamily siunitx} is active (and \emph{not} in \ConTeXt), then the following are equivalent:

\begin{LTXexample}[varwidth=true]
\begin{circuitikz}
   \draw (0,0) to[R, l=1<\kilo\ohm>] (2,0);
\end{circuitikz}
\end{LTXexample}	

\begin{LTXexample}[varwidth=true]
\begin{circuitikz}
   \draw (0,0) to[R, l=$\SI{1}{\kilo\ohm}$] (2,0);
\end{circuitikz}
\end{LTXexample}	

\begin{LTXexample}[varwidth=true]
\begin{circuitikz}
   \draw (0,0) to[R, i=1<\milli\ampere>] (2,0);
\end{circuitikz}
\end{LTXexample}	

\begin{LTXexample}[varwidth=true]
\begin{circuitikz}
   \draw (0,0) to[R, i=$\SI{1}{\milli\ampere}$] (2,0);
\end{circuitikz}
\end{LTXexample}	

\begin{LTXexample}[varwidth=true]
\begin{circuitikz}
   \draw (0,0) to[R, v=1<\volt>] (2,0);
\end{circuitikz}
\end{LTXexample}	

\begin{LTXexample}[varwidth=true]
\begin{circuitikz}
   \draw (0,0) to[R, v=$\SI{1}{\volt}$] (2,0);
\end{circuitikz}
\end{LTXexample}	



\subsection{Mirroring}

\begin{LTXexample}[varwidth=true]
\begin{circuitikz}
   \draw (0,0) to[pD] (2,0);
\end{circuitikz}
\end{LTXexample}	

\begin{LTXexample}[varwidth=true]
\begin{circuitikz}
   \draw (0,0) to[pD, mirror] (2,0);
\end{circuitikz}
\end{LTXexample}	

At the moment, placing labels and currents on mirrored bipoles works:
\begin{LTXexample}[varwidth=true]
\begin{circuitikz}
   \draw (0,0) to[ospst=T] (2,0);
\end{circuitikz}
\end{LTXexample}	

\begin{LTXexample}[varwidth=true]
\begin{circuitikz}
   \draw (0,0) to[ospst=T, mirror, i=$i_1$] (2,0);
\end{circuitikz}
\end{LTXexample}	

But voltages don't:

\begin{LTXexample}[varwidth=true]
\begin{circuitikz}
   \draw (0,0) to[ospst=T, mirror, v=v] (2,0);
\end{circuitikz}
\end{LTXexample}	

Sorry about that.

\subsection{Putting them together}
\begin{LTXexample}[varwidth=true]
\begin{circuitikz}
   \draw (0,0) to[R=1<\kilo\ohm>,
      i>_=1<\milli\ampere>, o-*] (3,0);
\end{circuitikz}
\end{LTXexample}

\begin{LTXexample}[varwidth=true]
\begin{circuitikz}
   \draw (0,0) to[D*, v=$v_D$,
      i=1<\milli\ampere>, o-*] (3,0);
\end{circuitikz}
\end{LTXexample}



\section{Not only bipoles}

Since only bipoles (but see section~\ref{sec:transasbip}) can be placed "along a line", components with more than two terminals are placed as nodes:
\begin{LTXexample}[varwidth=true]
\tikz \node[npn]  at (0,0) {};
\end{LTXexample}

\subsection{Anchors}

In order to allow connections with other components, all components define anchors. 

\subsubsection{Logical ports} All logical ports, except \textsc{not}, have two inputs and one output. They are called respectively \texttt{in 1}, \texttt{in 2}, \texttt{out}:

\begin{LTXexample}[varwidth=true]
\begin{circuitikz} \draw 
  (0,0) node[and port] (myand)  {}
  (myand.in 1) node[anchor=east] {1}
  (myand.in 2) node[anchor=east] {2}
  (myand.out) node[anchor=west] {3}
;\end{circuitikz}
\end{LTXexample}

\begin{LTXexample}[varwidth=true]
\begin{circuitikz} \draw 
  (0,2) node[and port] (myand1)  {}
  (0,0) node[and port] (myand2)  {}
  (2,1) node[xnor port] (myxnor)  {}
  (myand1.out) -| (myxnor.in 1)
  (myand2.out) -| (myxnor.in 2)
;\end{circuitikz}
\end{LTXexample}

In the case of \textsc{not}, there are only \texttt{in} and \texttt{out} (although for compatibility reasons \texttt{in 1} is still defined and equal to \texttt{in}):

\begin{LTXexample}[varwidth=true]
\begin{circuitikz} \draw 
  (1,0) node[not port] (not1)  {}
  (3,0) node[not port] (not2)  {}
  (0,0) -- (not1.in) 
  (not2.in) -- (not1.out) 
  ++(0,-1) node[ground] {} to[C] (not1.out) 
  (not2.out) -| (4,1) -| (0,0)
;\end{circuitikz}
\end{LTXexample}

\subsubsection{Transistors} For \textsc{nmos}, \textsc{pmos}, \textsc{nfet}, \textsc{nigfete}, \textsc{nigfetd}, \textsc{pfet}, \textsc{pigfete}, and \textsc{pigfetd}  transistors  one has \texttt{base}, \texttt{gate}, \texttt{source} and \texttt{drain} anchors (which can be abbreviated with \texttt{B}, \texttt{G}, \texttt{S} and \texttt{D}):

\begin{LTXexample}[varwidth=true]
\begin{circuitikz} \draw 
  (0,0) node[nmos] (mos)  {}
  (mos.base) node[anchor=west] {B}
  (mos.gate) node[anchor=east] {G}
  (mos.drain) node[anchor=south] {D}
  (mos.source) node[anchor=north] {S}
;\end{circuitikz}
\end{LTXexample}

\begin{LTXexample}[varwidth=true]
\begin{circuitikz} \draw 
  (0,0) node[pigfete] (pigfete)  {}
  (pigfete.B) node[anchor=west] {B}
  (pigfete.G) node[anchor=east] {G}
  (pigfete.D) node[anchor=south] {D}
  (pigfete.S) node[anchor=north] {S}
;\end{circuitikz}
\end{LTXexample}

Similarly \textsc{njfet} and \textsc{pjfet} have  \texttt{gate}, \texttt{source} and \texttt{drain} anchors (which can be abbreviated with  \texttt{G}, \texttt{S} and \texttt{D}):

\begin{LTXexample}[varwidth=true]
\begin{circuitikz} \draw 
  (0,0) node[pjfet] (pjfet)  {}
  (pjfet.G) node[anchor=east] {G}
  (pjfet.D) node[anchor=north] {D}
  (pjfet.S) node[anchor=south] {S}
;\end{circuitikz}
\end{LTXexample}

For \textsc{npn}, \textsc{pnp}, \textsc{nigbt}, and \textsc{pigbt} transistors the anchors are  \texttt{base}, \texttt{emitter} and \texttt{collector} anchors (which can be abbreviated with \texttt{B}, \texttt{E} and \texttt{C}):

\begin{LTXexample}[varwidth=true]
\begin{circuitikz} \draw 
  (0,0) node[npn] (npn)  {}
  (npn.base) node[anchor=east] {B}
  (npn.collector) node[anchor=south] {C}
  (npn.emitter) node[anchor=north] {E}
;\end{circuitikz}
\end{LTXexample}

\begin{LTXexample}[varwidth=true]
\begin{circuitikz} \draw 
  (0,0) node[pigbt] (pigbt)  {}
  (pigbt.B) node[anchor=east] {B}
  (pigbt.C) node[anchor=north] {C}
  (pigbt.E) node[anchor=south] {E}
;\end{circuitikz}
\end{LTXexample}

Here is one composite example (please notice that the \texttt{xscale=-1} style would also reflect the label of the transistors, so here a new node is added and its text is used, instead of that of \texttt{pnp1}):

\begin{LTXexample}[varwidth=true]
\begin{circuitikz} \draw 
  (0,0) node[pnp] (pnp2) {2}
  (pnp2.B) node[pnp, xscale=-1, anchor=B] (pnp1) {}
    (pnp1) node {1}
  (pnp1.C) node[npn, anchor=C] (npn1) {}
  (pnp2.C) node[npn, xscale=-1, anchor=C] (npn2) {}
  (pnp1.E) -- (pnp2.E)  (npn1.E) -- (npn2.E)
  (pnp1.B) node[circ] {} |- (pnp2.C) node[circ] {}
;\end{circuitikz}
\end{LTXexample}

Similarly, transistors and other components can be reflected vertically:
\begin{LTXexample}[varwidth=true]
\begin{circuitikz} \draw 
  (0,0) node[pigfete, yscale=-1] (pigfete)  {}
  (pigfete.B) node[anchor=west] {B}
  (pigfete.G) node[anchor=east] {G}
  (pigfete.D) node[anchor=north] {D}
  (pigfete.S) node[anchor=south] {S}
;\end{circuitikz}
\end{LTXexample}

\begin{LTXexample}
   \begin{circuitikz}
        \draw (0,2) 
            node[rground, yscale=-1] {} 
        to[R=$R_1$] (0,0) 
            node[sground] {};
    \end{circuitikz} 
\end{LTXexample}

\subsubsection{Other tripoles} When inserting a thrystor, a triac or a potentiometer, one needs to refer to the third node--gate (\texttt{gate} or \texttt{G}) for the former two; wiper (\texttt{wiper} or \texttt{W}) for the latter one. This is done by giving a name to the bipole:

\begin{LTXexample}[varwidth=true]
\begin{circuitikz} \draw 
  (0,0) to[Tr, n=TRI] (2,0) 
        to[pR, n=POT] (4,0);
  \draw[dashed] (TRI.G) -| (POT.wiper) 
;\end{circuitikz}
\end{LTXexample}

As for the switches:
\begin{LTXexample}[varwidth=true]
\begin{circuitikz} \draw 
  (0,0) node[spdt] (Sw) {}
  (Sw.in) node[left] {in}
  (Sw.out 1) node[right] {out 1}
  (Sw.out 2) node[right] {out 2}
;\end{circuitikz}
\end{LTXexample}
\begin{LTXexample}[varwidth=true]
\begin{circuitikz} \draw 
 (0,0) to[C] (1,0) to[toggle switch , n=Sw] (2.5,0) 
   -- (2.5,-1) to[battery1] (1.5,-1) to[R] (0,-1) -| (0,0)
  (Sw.out 2) -| (2.5, 1) to[R] (0,1) -- (0,0)
;\end{circuitikz}
\end{LTXexample}

And the mixer:
\begin{LTXexample}[varwidth=true]
\begin{circuitikz} \draw 
  (0,0) node[mixer] (mix) {}
  (mix.in 1) node[left] {in 1}
  (mix.in 2) node[below] {in 2}
  (mix.out) node[right] {out}
;\end{circuitikz}
\end{LTXexample}
\subsubsection{Operational amplifier} The op amp defines the inverting input (\texttt{-}), the non-inverting input (\texttt{+}) and the output (\texttt{out}) anchors:

\begin{LTXexample}[varwidth=true]
\begin{circuitikz} \draw 
  (0,0) node[op amp] (opamp) {}
  (opamp.+) node[left] {$v_+$}
  (opamp.-) node[left] {$v_-$}
  (opamp.out) node[right] {$v_o$}
;\end{circuitikz}
\end{LTXexample}

There are also two more anchors defined, \texttt{up} and \texttt{down}, for the power supplies:
\begin{LTXexample}[varwidth=true]
\begin{circuitikz} \draw 
  (0,0) node[op amp] (opamp) {}
  (opamp.+) node[left] {$v_+$}
  (opamp.-) node[left] {$v_-$}
  (opamp.out) node[right] {$v_o$}
  (opamp.down) node[ground] {}
  (opamp.up) ++ (0,.5) node[above] {\SI{12}{\volt}} 
     -- (opamp.up)
;\end{circuitikz}
\end{LTXexample}

The fully differential op amp defines two outputs:
\begin{LTXexample}[varwidth=true]
\begin{circuitikz} \draw 
  (0,0) node[fd op amp] (opamp) {}
  (opamp.+) node[left] {$v_+$}
  (opamp.-) node[left] {$v_-$}
  (opamp.out +) node[right] {out +}
  (opamp.out -) node[right] {out -}
  (opamp.down) node[ground] {}
;\end{circuitikz}
\end{LTXexample}

\subsubsection{Double bipoles} All the (few, actually) double bipoles/quadrupoles have
the four anchors, two for each port. The first port, to the left, is port \texttt{A}, having the anchors \texttt{A1} (up) and \texttt{A2} (down); same for port \texttt{B}. They also expose the \texttt{base} anchor, for labelling:

\begin{LTXexample}[varwidth=true]
\begin{circuitikz} \draw 
  (0,0) node[transformer] (T) {}
  (T.A1) node[anchor=east] {A1}
  (T.A2) node[anchor=east] {A2}
  (T.B1) node[anchor=west] {B1}
  (T.B2) node[anchor=west] {B2}
  (T.base) node{K}
;\end{circuitikz}
\end{LTXexample}

\begin{LTXexample}[varwidth=true]
\begin{circuitikz} \draw 
  (0,0) node[gyrator] (G) {}
  (G.A1) node[anchor=east] {A1}
  (G.A2) node[anchor=east] {A2}
  (G.B1) node[anchor=west] {B1}
  (G.B2) node[anchor=west] {B2}
  (G.base) node{K}
;\end{circuitikz}
\end{LTXexample}

\subsection{Transistor paths}\label{sec:transasbip}

For syntactical convenience transistors can be placed using the normal path notation used for bipoles. The transitor type can be specified by  simply adding a ``T'' (for transistor) in front of the node name of the transistor. It will be placed with the base/gate orthogonal to the direction of the path:
\begin{LTXexample}[varwidth=true]
\begin{circuitikz} \draw
  (0,0) node[njfet] {1}
  (-1,2) to[Tnjfet=2] (1,2) 
    to[Tnjfet=3, mirror] (3,2);
;\end{circuitikz}
\end{LTXexample}

Access to the gate and/or base nodes can be gained by naming the transistors with the \texttt{n} or \texttt{name} path style:
\begin{LTXexample}[varwidth=true]
\begin{circuitikz} \draw[yscale=1.1, xscale=.8]
  (2,4.5) -- (0,4.5) to[Tpmos, n=p1] (0,3) 
     to[Tnmos, n=n1] (0,1.5) 
     to[Tnmos, n=n2] (0,0) node[ground] {}
  (2,4.5) to[Tpmos,n=p2] (2,3) to[short, -*] (0,3)
  (p1.G) -- (n1.G) to[short, *-o] ($(n1.G)+(3,0)$)
  (n2.G) ++(2,0) node[circ] {} -| (p2.G)
  (n2.G) to[short, -o] ($(n2.G)+(3,0)$)
  (0,3) to[short, -o] (-1,3)
;\end{circuitikz}
\end{LTXexample}

The \texttt{name} property is available also for bipoles, although this is useful mostly for triac, potentiometer and thyristor (see~\ref{sec:othertrip}).

\section{Customization}

\subsection{Parameters}

Pretty much all Circui\TikZ\ relies heavily on \texttt{pgfkeys} for value handling and configuration. Indeed, at the beginning of \texttt{circuitikz.sty} a series of key definitions can be found that modify all the graphical characteristics of the package.

All can be varied using the \verb!\ctikzset! command, anywhere in the code.

\paragraph{Shape of the components} (on a per-component-class basis)
\begin{LTXexample}[varwidth=true]
\tikz \draw (0,0) to[R=1<\ohm>] (2,0); \par
\ctikzset{bipoles/resistor/height=.6}
\tikz \draw (0,0) to[R=1<\ohm>] (2,0);
\end{LTXexample}

\begin{LTXexample}[varwidth=true]
\tikz \draw (0,0) node[nand port] {}; \par
\ctikzset{tripoles/american nand port/input height=.2}
\ctikzset{tripoles/american nand port/port width=.2}
\tikz \draw (0,0) node[nand port] {};
\end{LTXexample}

\paragraph{Thickness of the lines} (globally)
\begin{LTXexample}[varwidth=true]
\tikz \draw (0,0) to[C=1<\farad>] (2,0); \par
\ctikzset{bipoles/thickness=1}
\tikz \draw (0,0) to[C=1<\farad>] (2,0);
\end{LTXexample}


\paragraph{Global properties} Of voltage and current
\begin{LTXexample}[varwidth=true]
\tikz \draw (0,0) to[R, v=1<\volt>] (2,0); \par
\ctikzset{voltage/distance from node=.1}
\tikz \draw (0,0) to[R, v=1<\volt>] (2,0);
\end{LTXexample}

\begin{LTXexample}[varwidth=true]
\tikz \draw (0,0) to[C, i=$\imath$] (2,0); \par
\ctikzset{current/distance = .2}
\tikz \draw (0,0) to[C, i=$\imath$] (2,0);
\end{LTXexample}

\noindent However, you can override the properties \verb!voltage/distance from node!\footnote{That is, how distant from the initial and final points of the path the arrow starts and ends.}, \verb!voltage/bump b!\footnote{Controlling how high the bump of the arrow is --- how curved it is.} and \verb!voltage/european label distance!\footnote{Controlling how distant from the bipole the voltage label will be.} on a per-component basis, in order to fine-tune the voltages:

\begin{LTXexample}[varwidth=true]
\tikz \draw (0,0) to[R, v=1<\volt>] (1.5,0) 
       to[C, v=2<\volt>] (3,0); \par
\ctikzset{bipoles/capacitor/voltage/%
     distance from node/.initial=.7}
\tikz \draw (0,0) to[R, v=1<\volt>] (1.5,0)
       to[C, v=2<\volt>] (3,0); \par
\end{LTXexample}

\noindent Admittedly, not all graphical properties have understandable names, but for the time it will have to do:
\begin{LTXexample}[varwidth=true]
\tikz \draw (0,0) node[xnor port] {};
\ctikzset{tripoles/american xnor port/aaa=.2}
\ctikzset{tripoles/american xnor port/bbb=.6} 
\tikz \draw (0,0) node[xnor port] {};
\end{LTXexample}

\subsection{Components size}
Perhaps the most important parameter is \verb!\circuitikzbasekey/bipoles/length!, which 
can be interpreted as the length of a resistor (including reasonable connections): all other lenghts are relative to this value. For instance:

\begin{LTXexample}[pos=t,varwidth=true]
\ctikzset{bipoles/length=1.4cm} 
\begin{circuitikz}[scale=1.2]\draw
  (0,0) node[anchor=east] {B}
        to[short, o-*] (1,0)
        to[R=20<\ohm>, *-*] (1,2)
        to[R=10<\ohm>, v=$v_x$] (3,2) -- (4,2)
        to[cI=$\frac{\si{\siemens}}{5} v_x$, *-*] (4,0) -- (3,0)
        to[R=5<\ohm>, *-*] (3,2)
  (3,0) -- (1,0)
  (1,2) to[short, -o] (0,2) node[anchor=east]{A}  
;\end{circuitikz}
\end{LTXexample}

\begin{LTXexample}[pos=t,varwidth=true]
\ctikzset{bipoles/length=.8cm} 
\begin{circuitikz}[scale=1.2]\draw
  (0,0) node[anchor=east] {B}
        to[short, o-*] (1,0)
        to[R=20<\ohm>, *-*] (1,2)
        to[R=10<\ohm>, v=$v_x$] (3,2) -- (4,2)
        to[cI=$\frac{\siemens}{5} v_x$, *-*] (4,0) -- (3,0)
        to[R=5<\ohm>, *-*] (3,2)
  (3,0) -- (1,0)
  (1,2) to[short, -o] (0,2) node[anchor=east]{A}  
;\end{circuitikz}
\end{LTXexample}

\subsection{Colors}

The color of the components is stored in the key \verb!\circuitikzbasekey/color!. Circui\TikZ\ tries to follow the color set in \TikZ, although sometimes it fails. If you change color in the picture, please do not use just the color name as a style, like \verb![red]!, but rather assign the style \verb![color=red]!.

Compare for instance
\begin{LTXexample}[varwidth=true]
\begin{circuitikz} \draw[red]
  (0,2) node[and port] (myand1)  {}
  (0,0) node[and port] (myand2)  {}
  (2,1) node[xnor port] (myxnor)  {}
  (myand1.out) -| (myxnor.in 1)
  (myand2.out) -| (myxnor.in 2)
;\end{circuitikz}
\end{LTXexample}

and

\begin{LTXexample}[varwidth=true]
\begin{circuitikz} \draw[color=red]
  (0,2) node[and port] (myand1)  {}
  (0,0) node[and port] (myand2)  {}
  (2,1) node[xnor port] (myxnor)  {}
  (myand1.out) -| (myxnor.in 1)
  (myand2.out) -| (myxnor.in 2)
;\end{circuitikz}
\end{LTXexample}

One can of course change the color \emph{in medias res}:
\begin{LTXexample}[pos=t, varwidth=true]
\begin{circuitikz} \draw 
  (0,0) node[pnp, color=blue] (pnp2) {}
  (pnp2.B) node[pnp, xscale=-1, anchor=B, color=brown] (pnp1) {}
  (pnp1.C) node[npn, anchor=C, color=green] (npn1) {}
  (pnp2.C) node[npn, xscale=-1, anchor=C, color=magenta] (npn2) {}
  (pnp1.E) -- (pnp2.E)  (npn1.E) -- (npn2.E)
  (pnp1.B) node[circ] {} |- (pnp2.C) node[circ] {}
;\end{circuitikz}
\end{LTXexample}

The all-in-one stream of bipoles poses some challanges, as only the actual body of the bipole, and not the connecting lines, will be rendered in the specified color. Also, please notice the curly braces around the \texttt{to}:
\begin{LTXexample}[varwidth=true]
\begin{circuitikz} \draw 
  (0,0) to[V=1<\volt>] (0,2)
      { to[R=1<\ohm>, color=red] (2,2) }
        to[C=1<\farad>] (2,0) -- (0,0)
;\end{circuitikz}
\end{LTXexample}

Which, for some bipoles, can be frustrating:
\begin{LTXexample}[varwidth=true]
\begin{circuitikz} \draw 
  (0,0){to[V=1<\volt>, color=red] (0,2) }
        to[R=1<\ohm>] (2,2) 
        to[C=1<\farad>] (2,0) -- (0,0)
;\end{circuitikz}
\end{LTXexample}

The only way out is to specify different paths:
\begin{LTXexample}[varwidth=true]
\begin{circuitikz} \draw[color=red]
  (0,0) to[V=1<\volt>, color=red] (0,2);
  \draw (0,2) to[R=1<\ohm>] (2,2) 
        to[C=1<\farad>] (2,0) -- (0,0)
;\end{circuitikz}
\end{LTXexample}

And yes: this is a bug and \emph{not} a feature\ldots

\section{FAQ}

\noindent Q: When using \verb!\tikzexternalize! I get the following error:
\begin{verbatim}
 ! Emergency stop.
\end{verbatim}

\noindent A: The \TikZ\ manual states:
\begin{quotation}
Furthermore, the library assumes that all \LaTeX\ pictures are ended
    with \verb!\end{tikzpicture}!.
\end{quotation}

Just substitute every occurrence of the environment \verb!circuitikz! with \verb!tikzpicture!. They are actually pretty much the same.

\bigskip

\noindent Q: How do I draw the voltage between two nodes?

\noindent A: Between any two nodes there is an open circuit!
\begin{LTXexample}[varwidth=true]
\begin{circuitikz} \draw
  node[ocirc] (A) at (0,0) {}
  node[ocirc] (B) at (2,1) {}
  (A) to[open, v=$v$] (B)
;\end{circuitikz}
\end{LTXexample}

\bigskip

\noindent Q: I cannot write \verb!to[R = $R_1=12V$]! nor \verb!to[ospst = open, 3s]!: I get errors.

\noindent A: It is a limitation of the \TikZ\ parser. Use \verb!to[R = $R_1{=}12V$]! and \verb!to[ospst = open{,} 3s]! instead.


\section{Examples}
\begin{LTXexample}[pos=t,varwidth=true]
\begin{circuitikz}[scale=1.4]\draw
  (0,0) to[C, l=10<\micro\farad>] (0,2) -- (0,3)
        to[R, l=2.2<\kilo\ohm>] (4,3) -- (4,2)
        to[L, l=12<\milli\henry>, i=$i_1$] (4,0) -- (0,0)
  (4,2) { to[D*, *-*, color=red] (2,0) }
  (0,2) to[R, l=1<\kilo\ohm>, *-] (2,2) 
        to[cV, v=$\SI{.3}{\kilo\ohm} i_1$] (4,2)
  (2,0) to[I, i=1<\milli\ampere>, -*] (2,2) 
;\end{circuitikz}
\end{LTXexample}

\begin{LTXexample}[pos=t,varwidth=true]
\begin{circuitikz}[scale=1.2]\draw
  (0,0) node[ground] {}
        to[V=$e(t)$, *-*] (0,2) to[C=4<\nano\farad>] (2,2)
        to[R, l_=.25<\kilo\ohm>, *-*] (2,0)
  (2,2) to[R=1<\kilo\ohm>] (4,2)
        to[C, l_=2<\nano\farad>, *-*] (4,0)
  (5,0) to[I, i_=$a(t)$, -*] (5,2) -- (4,2)
  (0,0) -- (5,0)
  (0,2) -- (0,3) to[L, l=2<\milli\henry>] (5,3) -- (5,2)
 
 {[anchor=south east] (0,2) node {1} (2,2) node {2} (4,2) node {3}}
;\end{circuitikz}
\end{LTXexample}

\begin{LTXexample}[pos=t,varwidth=true]
\begin{circuitikz}[scale=1.2]\draw
  (0,0) node[anchor=east] {B}
        to[short, o-*] (1,0)
        to[R=20<\ohm>, *-*] (1,2)
        to[R=10<\ohm>, v=$v_x$] (3,2) -- (4,2)
        to[cI=$\frac{\siemens}{5} v_x$, *-*] (4,0) -- (3,0)
        to[R=5<\ohm>, *-*] (3,2)
  (3,0) -- (1,0)
  (1,2) to[short, -o] (0,2) node[anchor=east]{A}  
;\end{circuitikz}
\end{LTXexample}
 
\begin{LTXexample}[pos=t,varwidth=true]
\begin{circuitikz}[scale=1]\draw
	(0,0) node[transformer] (T) {}
	(T.B2) to[pD] ($(T.B2)+(2,0)$) -| (3.5, -1)
	(T.B1) to[pD] ($(T.B1)+(2,0)$)  -| (3.5, -1)
;\end{circuitikz}
\end{LTXexample}


\begin{LTXexample}[pos=t,varwidth=true]
\begin{circuitikz}[scale=1]\draw
	(5,.5) node [op amp] (opamp) {}
	(0,0) node [left] {$U_{we}$} to [R, l=$R_d$, o-*] (2,0)
	to [R, l=$R_d$, *-*] (opamp.+)
	to [C, l_=$C_{d2}$, *-] ($(opamp.+)+(0,-2)$) node [ground] {}
	(opamp.out) |- (3.5,2) to [C, l_=$C_{d1}$, *-] (2,2) to [short] (2,0)
	(opamp.-) -| (3.5,2)
	(opamp.out) to [short, *-o] (7,.5) node [right] {$U_{wy}$}
;\end{circuitikz}
\end{LTXexample}
 
\begin{LTXexample}[pos=t,varwidth=true]\label{ex:compatibility}
\begin{circuitikz}[scale=1.2, american]\draw
  (0,2) to[I=1<\milli\ampere>] (2,2)
        to[R, l_=2<\kilo\ohm>, *-*] (0,0)
        to[R, l_=2<\kilo\ohm>] (2,0)
        to[V, v_=2<\volt>] (2,2)
        to[cspst, l=$t_0$] (4,2) -- (4,1.5)
        to [generic, i=$i_1$, v=$v_1$] (4,-.5) -- (4,-1.5)
  (0,2) -- (0,-1.5) to[V, v_=4<\volt>] (2,-1.5)
        to [R, l=1<\kilo\ohm>] (4,-1.5);

   \begin{scope}[xshift=6.5cm, yshift=.5cm]
    \draw [->] (-2,0) -- (2.5,0) node[anchor=west] {$v_1/\volt$};
    \draw [->] (0,-2) -- (0,2) node[anchor=west] {$i_1/\SI{}{\milli\ampere}$} ;
    \draw (-1,0) node[anchor=north] {-2} (1,0) node[anchor=south] {2}
          (0,1) node[anchor=west] {4} (0,-1) node[anchor=east] {-4} 
          (2,0) node[anchor=north west] {4}
          (-1.5,0) node[anchor=south east] {-3};
    \draw [thick] (-2,-1) -- (-1,1) -- (1,-1) -- (2,0) -- (2.5,.5);
    \draw [dotted] (-1,1) -- (-1,0) (1,-1) -- (1,0) 
          (-1,1) -- (0,1) (1,-1) -- (0,-1);
   \end{scope}  
\end{circuitikz}
\end{LTXexample}

\begin{tabular}{l}uit	\fbox{}
	\\
	\begin{lstlisting}
\usepackage{tikz}
\usetikzlibrary{circuits.ee.IEC}
\usetikzlibrary{positioning}

\usepackage[compatibility]{circuitikz}
\ctikzset{bipoles/length=.9cm}

\begin{document}
 \begin{tikzpicture}[circuit ee IEC]
  \draw (0,0) to [resistor={name=R}] (0,2)
	to[diode={name=D}] (3,2);
  \draw (0,0) to[*R=$R_1$] (1.5,0) to[*Tnpn] (3,0)
    to[*D](3,2);
 \end{tikzpicture}
\end{document}
	\end{lstlisting}
\end{tabular}

\section{Revision history}

\begin{itemize}
\item[\itshape version 0.3.1] (?)
	\begin{enumerate}
    	\item typo in documentation
	\item fixed an error in the angle computation in voltage and current routines
	\item fixed problem with label size when scaling a tikz picture
	\item added gas filled surge arrester
	\item added compatibility option to work with Tikz's own circuit library
    \end{enumerate}
\item[\itshape version 0.3.0] (20121229)
	\begin{enumerate}
	\item fixed gate node for a few transistors
    \item added mixer
    \item added fully differential op amp (by Kristofer M.~Monisit)
    \item now general settings for the drawing of voltage can be overridden for specific components
    \item made arrows more homogeneous (either the current/voltage one, or latex' by pgf)
    \item added the single battery cell
    \item added fuse and asymmetric fuse
    \item added toggle switch
    \item added varistor, photoresistor, thermocouple, push button
    \item added thermistor, thermistor ptc, thermistor ptc
    \item fixed misalignment of voltage label in vertical bipoles with names
    \item added isfet
    \item added noiseless, protective, chassis, signal and reference grounds (Luigi Luigi «Liverpool»)
    \end{enumerate}
\item[\itshape version 0.2.4]  (20110911).
	\begin{enumerate}
	\item added square voltage source (contributed by Alistair Kwan)
	\item added buffer and plain amplifier (contributed by Danilo Piazzalunga)
	\item added squid and barrier (contributed by  Cor Molenaar)
	\item added antenna and transmission line symbols contributed by Leonardo Azzinnari
	\item added the changeover switch spdt (suggestion of Fabio Maria Antoniali)
	\item rename of context.tex and context.pdf (thanks to Karl Berry)
	\item updated the email address
    \item in documentation, fixed wrong (non-standard) labelling of the axis in an example (thanks to prof. Claudio Beccaria)
    \item fixed scaling inconsistencies in quadrupoles    
    \item fixed division by zero error on certain vertical paths
    \item introduced options straighlabels, rotatelabels, smartlabels
    \end{enumerate}
\item[\itshape version 0.2.3]  (20091118).
   \begin{enumerate}
	\item fixed compatibility problem with label option from tikz
	\item Fixed resizing problem for shape ground
	\item Variable capacitor
	\item polarized capacitor
	\item ConTeXt support (read the manual!)
	\item nfet, nigfete, nigfetd, pfet, pigfete, pigfetd (contribution of Clemens Helfmeier and Theodor 
Borsche)
	\item njfet, pjfet (contribution of Danilo Piazzalunga)
	\item pigbt, nigbt
	\item \emph{backward incompatibility} potentiometer is now the standard resistor-with-arrow-in-the-middle; the old potentiometer is now known as variable resistor (or vR), similarly to variable inductor and variable capacitor
	\item triac, thyristor, memristor
	\item new property "name" for bipoles
	\item fixed voltage problem for batteries in american voltage mode
	\item european logic gates
	\item \emph{backward incompatibility} new american standard inductor. Old american inductor now called "cute inductor"
	\item \emph{backward incompatibility} transformer now linked with the chosen type of inductor, and version with core, too. Similarly for variable inductor
	\item \emph{backward incompatibility} styles for selecting shape variants now end are in the plural to avoid conflict with paths
	\item new placing option for some tripoles (mostly transistors)
	\item mirror path style
\end{enumerate}
	
\item[\itshape version 0.2.2]  (20090520).
   \begin{enumerate}
		\item Added the shape for lamps.
		\item Added options \texttt{europeanresistor}, \texttt{europeaninductor}, \texttt{americanresistor} and \texttt{americaninductor}, with corresponding styles.
		\item \textbf{Fixed}: error in transistor arrow positioning and direction under negative \texttt{xscale} and \texttt{yscale}.
   \end{enumerate}   
   
   
\item[\itshape version 0.2.1]  (20090503).
   \begin{enumerate}
		\item Op-amps added.
		\item Added options \texttt{arrowmos} and \texttt{noarrowmos}.
   \end{enumerate}   


\item[\itshape version 0.2] First public release on CTAN (20090417).
   \begin{enumerate}
		\item \textbf{Backward incompatibility}: labels ending with \texttt{:}\textit{angle} are not parsed for positioning anymore.
		\item Full use of \TikZ\ keyval features.
		\item White background is not filled anymore: now the network can be drawn on a background picture as well.
		\item Several new components added (logical ports, transistors, double bipoles, \ldots).
		\item Color support.
		\item Integration with {\ttfamily siunitx}.
		\item Voltage, american style.
		\item Better code, perhaps. General cleanup at the very least.
   \end{enumerate}
\item[\itshape version 0.1] First public release (2007).
\end{itemize}


\printindex


\end{document}



